\documentclass[12pt,oneside]{fithesis2}
\usepackage[slovak]{babel}       % Multilingual support
\usepackage[utf8]{inputenc}       % UTF-8 encoding
\usepackage[T1]{fontenc}          % T1 font encoding
\usepackage[                      % A sans serif font that blends well with Palatino
  scaled=0.86
]{berasans}
\usepackage[                      % Clickable links
  plainpages = false,               % We have multiple page numberings
  pdfpagelabels                     % Generate pdf page labels
]{hyperref}
\usepackage{listings}             % Include the listings-package

\usepackage{caption}
\captionsetup[lstlisting]{position=bottom}

\thesislang{sk}                   % The language of the thesis
\thesistitle{Portál pre písanie a vyhodnocovanie diktátov}       % The title of the thesis
\thesissubtitle{Diplomová práca}  % The type of the thesis
\thesisstudent{Bc. Jakub Rumanovský}          % Your name
\thesiswoman{false}                % Your gender
\thesisfaculty{fi}                % Your faculty
\thesisyear{Jeseň \the\year}     % The academic term of your thesis defense
\thesisadvisor{Mgr. Marek Grác, Ph.D.}   % Your advisor

\begin{document}
  \FrontMatter                    % The front matter
    \ThesisTitlePage                % The title page
    \begin{ThesisDeclaration}       % The declaration
      \DeclarationText
      \AdvisorName
    \end{ThesisDeclaration}
    \begin{ThesisThanks}            % The acknowledgements (optional)
      Ďakujem vedúcemu práce Mgr. Marekovi Grácovi, Ph.D. za vedenie, trpezlivosť a rady pri písaní diplomovej
       práce a mojej rodine, kamarátom a priateľke za podporu.
    \end{ThesisThanks}
    \begin{ThesisAbstract}          % The abstract
      Úlohou diplomovej práce je vytvorenie diktátového webového systému určeného predovšetkým žiakom základných
      a stredných škôl a ich učiteľom. Zahŕňa to taktiež vytvorenie samostatného modulu vrámci aplikácie, ktorý
      bude verejne prístupný skrz rozhranie. To bude zabezpečovať opravu diktátu a bude použiteľné aj pre
      prípadné iné systémy.
    \end{ThesisAbstract}
    \begin{ThesisAbstracten}          % The abstract

      The aim of this Master thesis is to create a dictate web system, which can be used mainly by pupils of
      elementary school and high school and their teachers. It also contains creation of an independent module
      inside the app, that will be public and will correct the dictate based on the input text using proper
      endpoint. This module will also be reusable for other systems.
    \end{ThesisAbstracten}
    \begin{ThesisKeyWords}          % The keywords
      Java EE, dictate, trainer, grammar correction, correction module
    \end{ThesisKeyWords}
    \tableofcontents                % The table of contents
%   \listoftables                   % The list of tables (optional)
%   \listoffigures                  % The list of figures (optional)
  
  \MainMatter                     % The main matter
    \chapter{Úvod}          % Chapters
  
  \par Informačné technológie sa v súčasnosti dostávajú do stále väčšieho množstva odvetví ľudskej činnosti. Aj
do takých, v akých by sme ich ešte pred pár rokmi nečakali. A v mnohých majú nepopierateľne obrovský potenciál
rozvoja. Jedným z príkladov takéhoto odvetvia je e-learning, alebo slovensky vzdelávanie prostredníctvom
informačných technológií. V posledných rokoch došlo k rapídnemu rozvoju tejto oblasti. Prispeli k tomu
      nemalou mierou aj niektoré renomované univerzity \cite{crimson15}, ktoré poskytujú svoje kurzy zdarma alebo za menší poplatok. Celá táto evolúcia vzdelávania poskytuje doteraz netušené možnosti a to skutočne pre každého kto má záujem. Je dokonca možné absolvovať vopred vybrané kurzy a získať certifikát, ktorý si potom možno uviesť ako doplnkové vzdelanie v životopise a preukázať tak svoju kompetenciu v určitom odvetví alebo činnosti. Z horeuvedených príkladov je vidno, že e-learning v akejkoľvek forme má zmysel aj potenciál a prináša nesporné výhody pre žiakov, študentov aj vyučujúcich. 
  \par K dôležitým schopnostiam každého človeka, či už v profesionálnom alebo súkromnom živote, patrí písomný prejav. Jeden z jeho najpodstatnejších aspektov je správna gramatika a pravopis. Tie sa žiaci učia na školách od najútlejšieho veku a osvojovanie gramatiky prebieha už niekoľko desaťročí rovnakým spôsobom. Ku klasickým nástrojom na trénovanie správneho pravopisu patrí písanie diktátov. Do tejto oblasti moderné technológie ešte celkom nestihli preniknúť. Diktát sa píše na papier a následne je opravovaný a hodnotený učiteľom. Samozrejme už v súčasnosti existujú viac či menej kvalitné alternatívy (viď kapitola 2). 
  \par Kvôli neexistencii dostačujúcej aplikácie pre trénovanie diktátov za pomoci počítača, je v následujúcom texte navrhnutý webový portál, ktorý slúži ako alternatíva k testovaniu gramatiky v škole. Žiak sa môže kedykoľvek, či už na popud učiteľa alebo z vlastnej vôle, otestovať v písaní. Aplikácia si kladie za cieľ rozšíriť písanie a korekciu diktátov, ktorá sa momentálne píše žiakmi a opravuje učiteľmi, o výhody, ktoré ponúkajú informačné technológie. Konkrétne medzi ne patria rýchlejšia - a hlavne automatická - oprava chýb, možnosť ukladania a následného analyzovania chýb v širšom kontexte (či už v prípade konkrétneho študenta alebo prípadne celej triedy) a v neposlednom rade tiež zobrazenie týchto chýb v štatistike.
  \par Portál má už v dobe písania tejto práce niekoľko záujemcov, ktorí sa podujmú na funkčnom testovaní. Jednak sú to vybrané základné školy a jednak sa bude systém využívať na Pedagogickej fakulte Masarykovej univerzity pri výuke budúcich pedagógov.
    
    \chapter{Analýza systému}
       \section{Diktát}
	\par Diktát je podľa Prášilovej\cite{prasilova14} jeden z mnohých druhov pravopisných cvičení. Je to veľmi špecifická možnosť, ako skontrolovať u žiaka úroveň jeho pravopisného prejavu. Je to podľa nej však i prostriedok cvičebný, pretože sa z jeho pomocou upevňuje prebraté učivo. 
	\par Diktát je vo väčšine prípadov diktovaný učiteľom. Toto diktovanie má taktiež svoje zásady a predpisy. Podľa Hausera\cite{hauser94} sa riadi nasledovnými pravidlami:
	      \begin{enumerate}
	\item Učiteľ najprv prečíta celý text (a podľa potreby vysvetlí menej užívané výrazy)
	\item Diktuje po vetách. Najprv prečíta celú vetu a potom diktuje po častiach. Tempo žiakov prispôsobí veku žiakov a ich výkonnosti
	\item Diktuje presne a zreteľne, dodržuje pravidlá spisovnej výslovnosti, nezdôrazňuje nadmerne pauzy ani i/y s/z atď.
	\item Pri diktovaní sleduje učiteľ ako žiaci píšu. Stojí pred žiakmi a neprechádza sa po triede.
	\item Nakoniec prečíta učiteľ ešte raz pomaly celý text.
\end{enumerate}

\par Vybrané pravidlá, ktoré dávajú zmysel aj pri precvičovaní diktátu na internete, sú uplatnené aj pri vývoji portálu.
	

       \section{Existujúce riešenia}       
       		\subsection{Štruktúra existujúcich riešení}
       				\par Ako bolo zmienené vyššie, pri výuke gramatiky hrá ešte stále významnú úlohu zbierky diktátov, z ktorých je potom diktát diktovaný učiteľom v triede počas vyučovania. Taktiež existuje veľká skupina cvičení zadarmo dostupných na internete. Prášilová\cite{prasilova14} ich však vo svojej diplomovej práci nazýva skôr ako diktátmi doplňovacími cvičeniami. Výhodou takejto formy je podľa nej okamžité vyhodnotenie, ktoré však nemusí byť príliš dobrý motivačný prvok. V svojej diplomovej práci tiež uvádza, že by doplňovacie cvičenia mali slúžiť hlavne k precvičovaniu, nie ku klasifikácii. Ďalšou skupinou existujúcich riešení sú audiodiktáty, ktoré sú postavené na rovnakých textoch ako zbierky diktátov. Nevýhodou týchto diktátov sú technické problémy, ku ktorým môže dochádzať počas diktovania (napr. šum, zaseknutie pásky, preskakovanie). Ku komplikáciám môže podľa Prášilovej tiež dochádzať vo chvíli keď žiak zle rozumel, chce sa spýtať, tým sa dostane do sklzu a stráca pozornosť. Hlavnú nevýhodu však vidí v nemožnosti prispôsobiť tempo danej skupine. Výhodu potom vidí v zvýšenej pozornosti, ktorú môže učiteľ žiakom venovať namiesto čítania textu.
		\par V súčasnosti sa na internete objavujú aj celé portály zamerané na precvičovanie gramatických javov. Jedným z zo zástupcov je portál \textit{pravopisne.cz}, ktorý ponúka širokú paletu rôznych možností zahŕňajúcu doplňovačky, gramatickú teóriu, hry na precvičovanie pravopisu a gramatických javov ako aj tzv. audiodiktáty. Tieto však fungujú tak, že si žiak zvukový súbor cez prehliadač spustí a píše na papier a potom ho porovná so správnym riešením dostupným na stránke pri skončení cvičenia. Ďalším portálom dostupným na internete je web \textit{umimcesky.cz}. Na tomto portáli si žiak precvičuje gramatické javy pomocou testových otázok (väčšinou typu vyber správnu možnosť) a po označení odpovede sa zobrazí, či bola správna a ak áno, prečo boli ostatné nesprávne.
		
       		\subsection{Problémy v existujúcich riešeniach}
       		\subsection{Výhody a nevýhody existujúcich riešení}
       \section{Navrhovaný systém}
       		\subsection{Špecifikácia požiadaviek} \label{spec-app}
\par Požiadavka je niečo čo by malo byť implementované. Požiadavky sa delia na funkčné a nefunkčné. Funkčné požiadavky (FP) určujú aké chovanie bude systém bude systém ponúkať. Nefunkčné požiadavky (NFP) špecifikujú vlastnosti alebo obmedzujúce podmienky daného systému. Požiadavky sú základom všetkých systémov, sú v podstate vyjadrením toho, čo by mal systém robiť, v žiadnom prípade však toho, ako by to mal robiť\cite{arlow11}.
       		
\bigskip
     \textbf{NFP1 Na vývoj systému bude použitá platforma Java EE}
     \par Je definovaná konkrétna platforma, s použitím ktorej bude portál vyvíjaný, pričom použité technológie vrámci platformy nie sú vopred dané a ich výber bude špecifikovaný neskôr v texte.
     
\bigskip
     \textbf{NFP2 Jednoduchá rozšíriteľnosť}      		
	\par Webový systém na komplexnú prácu s diktátmi by mal byť ľahko rozšíriteľný o novú funkcionalitu, najmä ďalej zmienený samostatný modul určený na opravu používateľom napísaného textu.
	
\bigskip
     \textbf{FP1 Používateľské role}  
     \par V systéme budú vytvorené nasledujúce používateľské role s rôznou úrovňou oprávnení:
\begin{itemize}
	\item ADMINISTRATOR - používateľská rola bude slúžiť na správu systému. Jeho hlavnou úlohou je pridávať používateľov role TEACHER (učiteľ) a priradiť im skupiny používateľov do ich správy. Budú mať tiež možnosť spravovať portál, jeho používateľov, diktáty a podobne.
	\item TEACHER - rola reprezentujúca učiteľa. Bude mať možnosť pridávať nový diktát, spravovať priradenú skupinu používateľov.
	\item STUDENT - bude mať možnosť listovať dostupnými diktátmi, trénovať diktát, vidieť vlastné štatistiky a upravovať vlastný profil.
\end{itemize}

\bigskip
     \textbf{FP2 Správa skupín používateľov}  
     \par Učiteľovi bude pridelená skupina, alebo skupiny používateľov (triedy vrámci školy), ktorú bude môcť spravovať a vidieť ich štatistiky chýb v diktátoch, počet pokusov a podobne.
     
\bigskip
     \textbf{FP3 Nahrávanie diktátov do systému}  
     \par V systéme bude možné nahrávať audiosúbor s diktátom a priradiť k nemu popisné informácie priamo v prostredí portálu.     

\bigskip
     \textbf{FP4 Používateľské štatistiky}  
     \par Bude možné určeným skupinám používateľov zobraziť základné štatistiky používateľov a diktátov.
     
\bigskip
     \textbf{FP5 Prihlasovanie používateľov pomocou mailu alebo sociálnych sietí}  
	\par Každý používateľ bude mať možnosť sa prihlásiť klasicky pomocou mailovej adresy, alebo sociálnych sietí.
	
\bigskip
     \textbf{FP6 Registrácia používateľov}
     \par Používateľ musí mať možnosť sa pri prvom vstupe na portál zaregistrovať zvoleným emailom a heslom.
     
\bigskip
     \textbf{FP7 Písanie diktátov a ich oprava} 
     \par Žiak si bude môcť napísať diktát a po jeho napísaní si ho nechať systémom automaticky opraviť.
     
\bigskip
     \textbf{FP8 Samostatný modul integrovaný s webovou aplikáciou určený na opravu diktátov} 
     \par Portál bude obsahovať samostatný modul slúžiaci na opravu diktátov, ktorý bude prístupný cez známe rozhranie. Prístupný bude komukoľvek, kto ho bude chcieť použiť. Rozhranie aj kód budú riadne zdokumentované a pripravené na rozširovanie.         

\bigskip
     \textbf{FP9 Implementácia pravidiel nájdených počítačovými lingvistami}      		
	\par Vrámci portálu budú implementované pravidlá nájdené počítačovými lingvistami, ktoré budú detekovať a pridávať zdôvodnenia pre vybrané jazykové javy v češtine.

       		\subsection{Výhody a nevýhody navrhovaného systému}
       		
    \chapter{Návrh a implementácia systému}
    
    \par Webový systém bude rozdelený na tri úplne oddelené časti.

\par Dátovú vrstvu, označovaná anglickým slovom backend, ktorá sa bude starať o ukladanie dát a navonok bude možné k nej pristupovať pomocou tzv. REST endpointov – známych adries, ktoré po obdržaní dát v správnom formáte v JSON vrátia požadované údaje v definovanej forme.
	\par Druhou časťou bude prezentačná vrstva, známa aj pod anglickým ekvivalentom frontend, ktorá bude komunikovať s vrstvou prístupu k dátam skrz REST rozhrania. Tieto budú zabezpečené aj na strane backendu aj na strane frontendu proti neoprávnenému prístupu neautorizovaného používateľa. Ten sa bude musieť na získanie oprávnenia zaregistrovať. Registrovať bude možné iba užívateľov role STUDENT (žiaci), učiteľ (rola TEACHER) bude musieť požiadať o vytvorenie účtu administrátora (rola ADMINISTRATOR). Overenie prístupu bude prebiehať zavolaním špeciálnej adresy, definovanej v jednej z následujúcich kapitol spolu s užívateľským menom a heslom a následným povolením/odmietnutím prístupu na základe týchto údajov. Prezentačná vrstva bude navrhnutá ako tzv. jednostránková aplikácia\footnote{Single Page Application, skrátene SPA}.
	    \par Pri implementácii systému bol kladený dôraz na použitie moderných technológií a štruktúra je navrhnutá tak, aby sa ktorákoľvek súčasť dala v prípade potreby rýchlo nahradiť inou bez nutnosti zásahu do iných vrstiev systému. V následujúcich sekciách sú postupne uvedené použité technológie podľa príslušnosti k vrstve systému.
	    \par Poslednou časťou bude samostatný modul nasadený spoločne s aplikáciou, slúžiaci na opravu chýb. Takisto ako k backendu je k nemu možné pristupovať skrz REST rozhranie. Jeho úlohou bude vo vstupnom texte od užívateľa odhaliť chyby, označiť ich, zozbierať o každej chybe čo najviac informácií a na základe týchto potom zaradiť chyby do kategórií (definovaných v prílohe \ref{typy-chyb}) a priradiť k ním zodpovedajúci popis. Rozhranie bude možné použiť aj aplikáciami tretích strán. Štruktúra modulu je do hĺbky rozobratá v kapitole \ref{modul-diktaty} a popis rozhrania vrátane správneho vstupu a výstupu je popísaný v prílohe \ref{navod_oprava}.
	
	\includegraphics[width=\linewidth, scale=0.5]{usecase.png}    
	
	\pagebreak
	   	\section{Rozdelenie aplikácie na moduly}
      	\par Systém je rozdelený na 6 modulov. Prvý s názvom \texttt{dictatetrainer-model} obsahuje triedy zabezpečujúce prístup k databáze, byznis vrstvu aplikácie a k nim prislúchajúce testy. Modul \texttt{dictatetrainer-resource} obsahuje verejne prístupné rozhrania spolu s ich testami, \texttt{dictatetrainer-corrector} je zvláštny modul zabezpečujúci opravu diktátu a poskytuje verejne prístupné rozhranie. Modul \texttt{dictatetrainer-int-tests}, ako je už z názvu patrné, združuje integračné testy systému. \texttt{dictatetrainer-resource-war} pozostáva s prezentačnej vrstvy aplikácie a nakoniec modul \texttt{dictatetrainer-ear} zabezpečuje správne zbalenie celého systému, oddeľuje závislosti do zvláštneho adresára a backend do zvláštneho archívu tak, aby bola štruktúra aplikácie čo najprehľadnejšia.
      	
    \section{Štruktúra aplikácie}
          \subsection{Vrstva prístupu k databáze}
      
      \par Aby bolo možné abstrahovať definíciu tabuliek a relácii medzi nimi od konkrétnej implementácie databázového systému (ako je napr. Postgres, MySQL, Oracle a pod.), je nutné použiť framework umožňujúci objektovo-relačné mapovanie (skrátene ORM) tabuliek v databáze na Java objekty, tzv. entity. Entita je trieda typicky reprezentujúca jednu tabuľku v databáze. Obsahuje atribúty, ktoré sú ekvivalentné stĺpcom v tabuľke, bezparametrický konštruktor a, ak chceme definovať aj vzťahy medzi entitami, musí obsahovať aj preťažené metódy \texttt{equals} a \texttt{hashCode}. Štandardný framework umožňujúci ORM je Java Persistence API (skrátene JPA). Je to vlastne sada rozhraní definujúcich ako by malo ORM fungovať. Existuje viacero implementácií JPA - ako napr. EclipseLink, OpenJPA alebo Hibernate. Aplikácia používa poslednú spomínanú knižnicu.
      
      \subsubsection{Entity}
      \par Navrhovaný webový systém pozostáva z piatich navzájom prepojených entít (OBRÁZOK). Ich zoznam je jedným z výstupov analýzy aplikácie. Sú to \texttt{Category}, \texttt{Dictate}, \texttt{User}, \texttt{Trial} a \texttt{Error}. Nižšie je uvedený krátky popis.
      \begin{itemize}
	\item \texttt{Category} reprezentuje kategóriu diktátu, teda gramatický jav, ktorý má diktát v prvom rade precvičiť. Napr. vybrané slová, veľké a malé písmená, spodobovanie atď. Kategórie definuje správca systému.
	\item \texttt{Dictate} reprezentuje samotný diktát. Obsahuje základné informácie o diktáte: meno, krátky popis diktátu, názov súboru, pod ktorým je uložený na serveri a odkaz na kategóriu, do ktorej diktát spadá. Keďže jedna z požiadaviek na systém bola možnosť nahrávať diktáty, obsahuje tiež odkaz na používateľa, ktorý diktát vytvoril. Entita \texttt{Dictate} tiež ukladá dáta popisujúce samotný zvukový súbor s diktátom, menovite časové značky koncov viet, počet opakovaní diktátu a jednotlivých viet, pauzy medzi vetami a textový prepis diktátu, podľa ktorého potom bude prebiehať oprava.
	\item Entita \texttt{User} nesie základné informácie o používateľovi. Pod tým sa rozumie časová známka registrácie používateľa, meno, adresa elektronickej pošty (ktorá je unikátna, aby bolo možné sa pomocou nej prihlásiť), heslo v zašifrovanom tvare, typ používateľa a pridelená rola v systéme. Pretože má byť aplikácia rozdelená na používateľskú časť a časť pre učiteľov, delí sa entita na dva podtypy: študent a učiteľ, ktoré sa líšia tým, akého je používateľ typu (USER resp. STUDENT) a aké role môže zastávať (rola ADMINISTRATOR a TEACHER pre učiteľa resp. STUDENT pre študenta). 
	\item Entita \texttt{Trial} obsahuje informácie o každom dokončenom diktáte. Pozostáva s neopraveného textu napísaného študentom, časovej známky vykonania pokusu, odkazu na trénovaný diktát a rovnako odkazu na trénujúceho študenta. Tieto informácie môžu byť použité predovšetkým za účelom vytvárania štatistických dát.
	\item Entita \texttt{Error} reprezentuje jednu chybu používateľa v diktáte. Chyba nesie jednak všetky popisné informácie o konkrétnej chybe (ktoré sú totožné s výstupom modulu \texttt{Corrector}, pretože ho k svojmu vytvoreniu využíva). K tomu obsahuje odkaz na entitu \texttt{Trial}, odkaz na diktát, v ktorom sa chyba vyskytla a odkaz na používateľa, ktorý ju urobil. Rovnako ako u predchádzajúcej entity, aj tieto dáta sú predovšetkým určené na účely štatistiky. 
		\end{itemize}
      \par 
	\subsubsection{Prístup k entitám}
	\par Entity je nutné nielen vytvoriť, ale k nim aj vedieť pristúpiť a vykonať s nimi základné operácie. Základnými operáciami rozumieme tzv. CRUD operácie\footnote{\textbf{C}reate = vytvorenie, \textbf{R}ead = čítanie, \textbf{U}pdate = aktualizácia, \textbf{D}elete = zmazanie}. Tieto operácie sú definované v špeciálnej triede vlastnej pre každú entitu, ktorá tvorí akýsi medzičlánok spájajúci entity a databázu. K základným operáciám je pre každú entitu pridaná možnosť filtrovania výsledkov na základe určených atribútov (resp. z databázového pohľadu stĺpcov tabuľky).
      \subsection{Bussiness vrstva aplikácie}
      
      \par Servisná vrstva aplikácie vytvára ďalší stupeň abstrakcie aplikácie. Spolupracuje s vrstvou prístupu k dátam a poskytuje svoje rozhranie vrstve nad ňou.
      \subsection{Vrstva rozhraní - REST}
         \subsubsection{Nahrávanie diktátov}
   \par Jednou zo zadaných požiadaviek na aplikáciu bolo umožniť učiteľom nahrávať diktáty pomocou užívateľského grafického rozhrania a pridať k nim pomocné dáta, ktoré budú diktát popisovať (počet opakovaní jednotlivých viet, počet opakovaní celého diktátu, dĺžka pauzy medzi vetami...). Ako je spomenuté v kapitole \ref{ukladanie}, samotný súbor s diktátom je uložený oddelene od metadát.
   \par Návrh tejto časti bol rozdelený na tri etapy - sprístupnenie zložky určenej pre ukladanie diktátov na aplikačnom serveri, vytvorenie REST rozhrania pristupujúcemu k tejto zložke a navrhnutie nahrávania diktátu v prezentačnej vrstve na strane klienta.
   \par Zložka sprístupnená učiteľom pre nahrávanie diktátov musí byť umiestnená mimo samotný kontext aplikácie. Hlavným dôvodom je nutnosť zachovania nahratých súborov v prípade opätovného nasadenia aplikácie kvôli vylepšeniam resp. oprave chýb.

      \subsection{Prezentačná vrstva}
      		\subsubsection{Angular JS JavaScript framework}
			\subsubsection{Asynchrónne volanie}
      		\subsubsection{CSS a Bootstrap}
      		\subsubsection{HTML5}
      		\pagebreak
	\section{Ukladanie dát popisujúcich diktát}
	\par Dátami popisujúcimi diktát rozumieme všetky dáta uložené spolu s diktátom, ktoré ho nejakým spôsobom definujú. Konkrétne sa jedná o textový prepis diktátu (tzv. transkript), značky koncov jednotlivých viet, východzí počet opakovaní prehrávania viet ako aj celého diktátu a dĺžka pauzy medzi jednotlivými vetami. V bakalárskej práci bol systém navrhnutý tak, že boli tieto informácie uložené vrámci mp3 súboru s diktátom \cite{rumanov12}. Tento prístup má však niekoľko nevýhod. Jednak je použitie zvukových formátov obmedzený na mp3, pretože sa informácie ukladajú do ID3v2 tagu\footnote{Do súboru mp3 je možné uložiť rôzne atribúty, nazývané tagy, ktoré obsahujú napr. názov interpreta, názov skladby, rok vydania atď.}. Druhá nevýhoda je, že tento návrh komplikuje prípadnú budúcu úpravu systému a spracovanie napr. pre účely štatistiky, pretože v prípade budovania štatistických informácií by bolo nutné najprv pristúpiť ku každému súboru zvlášť a získať z neho potrebné dáta. V neposlednom rade je nevýhodou neflexibilita takéhoto riešenia. Každý súbor s diktátom by totiž pred nahratím na server musel najprv integrovať všetky dáta pomocou špecialneho nástroja dostupného v systéme.
	\par Toto všetko sú dôvody, prečo bolo nakoniec rozhodnuté ukladať všetky popisné dáta v databáze spolu s autorom a názvom súboru s diktátom a zvukový záznam je uložiť zvlášť na serveri. Viac o uložení súboru v kapitole \ref{ukladanie}.
	        
	\section{Ukladanie audio súborov s diktátmi na server} \label{ukladanie}
      \par Existujú dve možnosti ako ukladať audio súbory a všeobecne akékoľvek dáta. Prvý prístup ukladá dáta priamo do databázy ako BLOB\footnote{dátový typ pre bližšie nešpecifikované binárne dáta v databáze}. Nevýhoda tejto alternatívy je v tom, že neúmerne zväčšuje databázu, spomaľuje dotazy a všeobecne degraduje jej rýchlosť.
      \par Druhá možnosť je ukladať súbory priamo na server a do databázy pridať iba odkaz, resp. meno súboru ako jeden stĺpec v tabuľke. Tento prístup je pri veľkých súboroch odporúčaný \cite{sof1}. Rozhodol som sa preto oddeliť súbory s diktátmi od databázy a ponechať tam iba odkazy. Viac o konkrétnom riešení s použitím databázy Postgres a aplikačného servera Wildfly je uvedené v kapitole \ref{nasadenie}. Nasadenie systému.
     
      \section{Zabezpečenie aplikácie}
      		\subsection{Zabezpečená komunikácia pomocou HTTPS} \label{https}
      		\par Základom bezpečnosti akéhokoľvek webového systému je zabezpečená komunikácia. Dáta posielané užívateľom cez sieť vrámci interakcie so systémom vrátane mena, hesla a iných citlivých informácií, by bez nej boli vystavené potenciálnemu riziku odcudzenia. Takýto útok sa nazýva Man in the middle attack\footnote{Typ útoku, kedy útočník napadne komunikáciu medzi užívateľom a serverom s cieľom ukradnúť citlivé dáta}. Je tomu však možno zabrániť zašifrovaním komunikácie s použitím protokolu HTTPS, ktorý je bezpečnou verziou HTTP. Cezeň sa posielajú dáta medzi internetovým prehliadačom a webovou stránkou, ku ktorej je prehliadač pripojený. 
      		\par HTTPS typicky využíva jeden z dvoch protokolov na šifrovanie komunikácie - TLS alebo SSL. Oba používajú tzv. asymetrickú šifru, ktorá pracuje s dvoma typmi kľúčov. Privátny kľúč je bezpečne uložený na webovom serveri a zabezpečuje šifrovanie posielaných dát. Verejný kľúč je distribuovaný komukoľvek, kto chce rozšifrovať informácie zašifrované pomocou privátneho kľúča. Verejný kľúč obdrží používateľ v SSL certifikáte webovej stránky pri prvotnom požiadavku o spojenie \cite{comodo15}.
      		\par Vo vyvíjanej aplikácii, ktorá je nasadená na službe je vynútený protokol HTTPS. Openshift ponúka svoj SSL certifikát, takže nie je nutné vytvárať vlastný. Je iba potrebné správne nastaviť \texttt{web.xml} a \texttt{jboss-web.xml} podľa dokumentácie \cite{openshift15}.
      		\pagebreak

      		\subsection{Basic autentifikácia a AngularJS}
      		\par Podľa špecifikácie aplikácie uvedenej v kapitole \ref{spec-app} má navrhnutý systém obsahovať administračnú časť pre pedagógov a používateľskú časť pre žiakov. Z tohoto dôvodu je nutné zabezpečiť riadenie prístupu k zdrojom. Ako riešenie je použitá Basic autentifikácia používateľa medzi klientom a serverom.
      		\par Podľa servera \texttt{spaghetti.io}\cite{spaghetti14} je Basic autentifikácia jedna z najrozšírenejších a najviac používaných autentifikačných metód. Aby bolo možné implementovať túto funkcionalitu, musí byť na strane klienta do každej požiadavky pridaná hlavička obsahujúca Base64\footnote{Číslovací systém založený na 64 rôznych znakoch využívajúci veľmi jednoduchý kódovací/dekódovací algoritmus. Neponúka jednosmernú šifrovaciu funkciu ako napr. SHA1} reťazec vo formáte \texttt{Authorization:Basic username:password}. Napriek tomu, že sa meno a heslo nachádza v nezašifrovanej podobe v hlavičke, je protokol bezpečný, vyžaduje však zabezpečenú komunikáciu medzi klientom a serverom, teda HTTPS (kapitola \ref{https}). 
      		\par Implementácia Basic autentifikácie na strane klienta vychádza z návodu od Watmorea\cite{watmore14}. Je realizovaná tak, že sa najprv odošle požiadavka na konkrétny zdroj s danou HTTP metódou. Pri odpovedi prezentačná vrstva overí, či požiadavka skončila úspešne, ak nie zobrazí chybové hlásenie. Ak je dotaz úspešný, uložia sa informácie jednak do globálnej premennej, aby boli prístupné ostatným stránkam resp. im prislúchajúcim skriptom a jednak do HTTP cookie\footnote{dáta získané od webovej stránky, uložené v prehliadači používateľa}. Toto riešenie bolo zvolené z toho dôvodu, aby sa pri náhodnom alebo cielenom obnovení stránky nestratili prihlasovacie dáta a užívateľ sa nemusel znova prihlasovať do systému. Odhlásenie používateľa znamená presmerovanie na prihlasovaciu stránku a zmazanie HTTP cookie s prihlasovacími údajmi
      		
      		\subsection{Zabezpečenie ciest na úrovni AngularJS frameworku}
      		\par Zabezpečenie prezentačnej vrstvy pred neoprávneným prístupom používateľa je rovnako dôležité ako zabezpečenie časti aplikácie starajúcej sa o prístup k dátam. Ak je používateľ neprihlásený, alebo prihlásený pod rolou, ktorá k uvedenej adrese nemá mať prístup, musí byť zamedzený a používateľ primerane oboznámený.
      		\par 
      		
      		\subsection{Autentifikácia pomocou sociálnych sietí}
      		\par 
      		
      		\subsection{Cross Domain Resource Sharing (CORS)}
      		\par Jedným z požiadaviek na aplikáciu (kap. \ref{spec-app}) je vytvoriť rozhranie prístupné ostatným doménam, ktoré bude schopné opraviť užívateľský text podľa zadaného originálu. Túto funkciu bude môcť v budúcnosti využívať napr. Informačný systém Masarykovej univerzity vo svojich projektoch. Keďže ide o prístup k dátam a zdrojom aplikácie zvonku, je nutné riešiť tzv. Cross Domain Resource Sharing\footnote{V preklade: Zdieľanie zdrojov naprieč doménami}.
      		\par CORS HTTP požiadavka zdrojovej stránky je definovaná ako požiadavka na dáta alebo informácie z inej domény ako je doména zdroja. Dáta môžu byť v akejkoľvek forme, napr. obrázky, css štýly, skripty atď. Z bezpečnostných dôvodov sú CORS požiadavky vo všetkých webových prehliadačoch implicitne zakázané. Explicitne je však možné prístup pre určité rozhrania a domény povoliť. Ku konkrétnej odpovedi servera sa pridajú hlavičky ktoré popisujú množinu domén a HTTP metód, ktorým je povolený prístup. HTTP metódy, ktoré môžu spôsobiť vedľajšie efekty na odosielané používateľské dáta (predovšetkým GET a POST), je najprv vykonaná tzv. predpožiadavka\footnote{angl. Preflight request}, ktorý pozdrží odoslanie odpovede servera do prehliadača až do okamihu, keď je povolený prístup na základe hlavičiek\cite{mozilla2015}. Správny formát hlavičiek znižuje bezpečnostné riziká spojené s CORS, predovšetkým CSRF útok\footnote{Cross-Site Request Forgery, v preklade: Falšovanie požiadaviek cudzími doménami je typ útoku, ktorý vykoná požiadavku z inej domény ako doména servera využívajúc aktuálne prihláseného používateľa}\cite{sof4}. Definícia hlavičiek spolu s dokumentáciou sa nachádza na stránkach pracovnej skupiny W3C\cite{w3c2014}.
      		\par Povolenie prístupu musí byť definovaná na strane servera. Z návrhu aplikácie vyplýva, že je nutné povoliť jediný zdroj slúžiaci na opravu diktátov, využívajúci metódu POST. Java ponúka hneď niekoľko možností.
      		\par Prvá možnosť je použitie filtru, ktorý implementuje rozhranie \texttt{ContainerResponseFilter}\cite{matei14}\cite{sof3}. Tento spôsob však povoľuje prístup pre všetky zdroje, preto je pre aplikáciu nevhodný. Druhé riešenie je pridať prístupové hlavičky priamo pri budovaní odpovede spolu s telom odpovede a návratovým kódom. Výhoda tohto riešenia je, že umožňuje vybrať podmnožinu dostupných zdrojov u ktorých bude povolený CORS. Bohužiaľ, tento postup nie je možné použiť pre HTTP metódu POST, pretože ak je požiadavka vykonaná z inej domény, hlavičky sa nepridajú a odpoveď skončí s chybou\cite{sof2}. 
      		\par Riešenie spĺňajúce obmedzenia aplikácie využíva externú knižnicu\cite{dzhuvinov15}, ktorá definuje prístupové hlavičky priamo v súbore \texttt{web.xml}. Tento postup jednak ponúka možnosť vybrať z dostupných zdrojov tie, ktoré budú povolené, a taktiež funguje s HTTP metódou POST\cite{sof2}.
\pagebreak

      \chapter{Modul na opravu diktátov} \label{modul-diktaty}
      
      \section{Návrh}
      \par Modul na opravu diktátov je nezávislý od ostatných modulov aplikácie a pozostáva zo značkovača vstupného používateľského textu a definície pravidiel. Prístup k modulu zabezpečuje verejné REST rozhranie. Vstupom je správny text diktátu z databázy a text vložený používateľom. Výstup vo formáte JSON obsahuje celkový počet chýb a list jednotlivých chýb zoradených podľa pozície chyby v diktáte (viac o definícii rozhrania, formáte vstupu a výstupu v prílohe \ref{navod_oprava}). 
      \par Opravu diktátu možno rozdeliť do niekoľkých fáz. Najprv sa vstupné reťazce porovnajú za pomoci knižnice \texttt{diff-match-patch}\cite{diffmatchpatch} a vytvorí sa označkovaný text. Z takto predspracovaného textu sa v ďalšom kroku vygeneruje pre každú chybu objekt typu \texttt{Mistake}. Tento objekt je určený na uchovanie dát popisujúcich jednotlivé chyby. Štruktúra atribútov vychádza z návrhu popísaného v bakalárskej práci Vojtěcha Škvařila \cite{skvaril14}. V nasledujúcom texte bude ako správny označený znak alebo slovo vyskytujúce sa v správnom texte diktátu z databázy a chybným znakom alebo slovom rozumieme podreťazec textu vloženého používateľom, ktorý sa nezhoduje so správnym znakom resp. slovom. Slovo je v tomto texte synonymum k výrazu token\footnote{Token je kategorizovaný blok textu, obyčajne pozostáva z nedeliteľných znakov} s výnimkou, že slovo môže pozostávať s dvoch tokenov, vlastného slova a interpunkčného znaku. Posledná fáza aplikuje definované pravidlá a na ich základe pridáva definíciu chyby, typ chyby a prioritu.
\par Každá inštancia chyby obsahuje nasledovné atribúty:
      \begin{itemize}
	\item \texttt{id} - jednoznačný identifikátor objektu
	\item \texttt{mistakeCharPosInWord} - pozícia chybného znaku, pri viacerých chybných znakoch označených za sebou ako jedna chyba vracia pozíciu prvého chybného znaku, v prípade nadbytočného znaku je pozícia 0, v prípade chýbajúceho -1.
	\item \texttt{correctChars} - správny znak resp. podreťazec. Ak sa jedná o nadbytočný znak, môže byť prázdny
	\item \texttt{writtenChars} - chybný znak resp. podreťazec označený ako jedna chyba. Ak sa jedná o chýbajúci znak, môže byť prázdny
	\item \texttt{correctWord} - správne slovo
	\item \texttt{writtenWord} - chybné slovo bez značiek označujúcich chybu
	\item \texttt{previousWord} - slovo bezprostredne predchádzajúce chybnému slovu
	\item \texttt{nextWord} - slovo bezprostredne nasledujúce po chybnom slove
	\item \texttt{wordPosition} - pozíciu chyby v diktáte, použiteľnú napr. pri výpise chýb konkrétneho slova. Pozícia je definovaná ako číslo tokenu vrámci diktátu, číslovaná od 1, 0 znamená nadbytočné slovo, -1 chýbajúce slovo
	\item \texttt{lemma} - základný tvar slova získaný z výstupu morfologického analyzátoru Majka\cite{majka}, ak je vo výstupe viac ako jeden tvar, berie sa vždy prvý v poradí
	\item \texttt{posTag} - značka popisujúca slovné druhy a iné morfologické(?) kategórie, rovnako získané z výstupu analyzátoru Majka. Popis značiek je dostupný na webovej stránke NLP FI\cite{nlpfi}, pričom znova platí, že ak je z výstupu dostupných viacero možností, je braná vždy prvá v poradí
	\item \texttt{sentence} - veta v ktorej sa vyskytla chyba. Ukladá sa veta z výstupu značkovača, teda s označenými chybami. Tento atribút je určený pre budúce využitie v prípade popisu kontextových chýb, ktoré sa v tejto práci neriešia
	\item \texttt{priority} - prioritu slova. Priorita je číslo od 1 do 10, pričom 10 je najväčšia priorita a získava sa spolu s typom chyby a jej popisom ako výstup definície pravidiel
	\item \texttt{mistakeType} - typ chyby, tzn. zaradenie chyby do určitej kategórie. Úplný zoznam možných typov chýb je dostupný v prílohe \ref{typy-chyb}.
	\item \texttt{mistakeDescription} - popis chyby na základe daných pravidiel.
\end{itemize}
      \par Celý modul je navrhnutý s dôrazom na modifikovateľnosť a nahraditeľnosť jednotlivých častí. Stačí vymeniť implementácie daných Java rozhraní. Modul je používaný aj samotnou navrhovanou aplikáciou, kde využíva objekty typu \texttt{Mistake} z popisovaného modulu a ukladá ich do databázy spolu s informáciami o používateľovi, diktáte a prístupe k diktátu ako objekt typu \texttt{Error}. Návrh bol zvolený z toho dôvodu, aby bolo rozhranie opravy diktátu čo najjednoduchšie použiteľné aplikáciami tretích strán (preto má objekt typu \texttt{Mistake} iba atribúty, ktoré je možné vyčítať z daných vstupných dát). Zamýšľané použitie objektu typu \texttt{Error} je na generovanie štatistických dát o jednotlivých používateľoch a diktátoch.
      
      \section{Značkovanie chýb}
      \par Formát značkovania vychádza z návrhu prezentovanom v mojej bakalárskej práci \cite{rumanov12}. Je definovaný nasledujúcim spôsobom:
      \begin{itemize}
	\item Ak sa v slove vyskytuje chyba, prípadne je chybné celé slovo či slová, označí sa znak, reťazec, slovo alebo slová špeciálnymi značkami definovanými nižšie
	\item Správny znak resp. reťazec je uzavretý do párových zátvoriek \textit{(} a \textit{)}. Chybný znak alebo reťazec je označený zátvorkami \textit{<} a \textit{>} nasledujúcimi bezprostredne po správnej variante (napr. \textit{r(y)<i>ba})
	\item Chýbajúci znak alebo reťazec znakov je označený zátvorkami \textit{(} a \textit{)} a bezprostredne za nimi pokračuje dané slovo (napr. \textit{strun(n)ý}) resp. sa môže nachádzať koniec slova (napr. \textit{košil(e)}), čo sa väčšinou vyskytuje pri preklepoch.
	\item Nadbytočný znak v slove je označený zátvorkami \textit{<} a \textit{>} a bezprostredne za nimi slovo pokračuje (napr. \textit{ran<n>ý}) resp. sa môže nachádzať koniec slova (napr. \textit{košile<e>}), čo sa väčšinou vyskytuje pri preklepoch.
	\item Chýbajúce slovo alebo slová sú celé uzavreté do zátvoriek a za ním nenasledujú párové zátvorky \textit{<} a \textit{>}, ale koniec slova, prípadne interpunkčný znak\footnote{Interpunkčným znakom rozumieme znak označujúci koniec vety (\textit{. ! ?}) a špeciálne znaky (\textit{" , ; :})} (napr. \textit{Stěhuje se (z domu).})
	\item Nadbytočné slovo resp. slová sú uzavreté do zátvoriek \textit{<} a \textit{>} a za uzavieracou zátvorkou nasleduje koniec slova, prípadne interpunkčný znak (napr. \textit{Mává <s> kapesníkem.})
	\end{itemize}
	\par Slovo je v mojej práci definované ako reťazec znakov v texte oddelený medzerami. Slovo vo veľkej väčšine obsahuje jeden token. Existujú dve výnimky. Ak je slovo na konci vety, (Veta je definovaná ako reťazec slov ukončených interpunkčným znakom) obsahuje aj interpunkčný znak, ktorý je súčasťou slova. Ak sa v užívateľskom vstupe objaví chýbajúca alebo nadbytočná medzera, sú tokeny naľavo a napravo od chyby kvôli jednoduchšiemu spracovaniu brané ako jedno slovo. Chybný znak, reťazec znakov, chýbajúce a nadbytočné slová sú získané zo vstupu zadaného užívateľom. Správny znak alebo reťazec znakov je daný podľa prepisu diktátu uloženého v databáze aplikácie. Z označkovaného textu je pri každom výskyte chyby vytvorený nový objekt typu \texttt{Mistake}. Jeho štruktúra je popísaná vyššie.
	
      \section{Spracovanie chýb podľa definovaných pravidiel}
     	  \par Po spracovaní textu a vytvorení \texttt{Mistake} objektu pre každú chybu spracovanie prechádza do ďalšej fázy. Prostredníctvom dát uložených v objekte a popisujúcich chybu sa ju systém pokúsi čo najpresnejšie zaradiť do jednej z kategórií (popísaných v prílohe \ref{typy-chyb}) a priradiť mu odpovedajúce vysvetlenie a prioritu. Priorita je váha, ktorú majú jednotlivé typy chýb a je možné ju napr. použiť pri výpočte výslednej známky.
     	  \par Implementácia pravidiel, ktoré dokážu chybu správne zaradiť, bola jednou z požiadaviek na výslednú aplikáciu. Zoznamy vyskytujúce sa vrámci definície pravidiel sú umiestnené v samostatnej statickej triede. Statická trieda bola zvolená z toho dôvodu, aby sa pri každom prístupe nevytvárala nová inštancia triedy obsahujúcej zoznamy. Trieda s pravidlami k nim potom pristupuje priamo pomocou názvu triedy a názvu zoznamu.
     	  \par Pravidlá sú prepísané z pseudokódu, ktorý bol súčasťou bakalárskej práce Vojtěcha Škvařila\cite{skvaril14}. Bolo v nich však vykonaných zopár zmien kvôli zlepšeniu fungovania pravidiel a ich presnosti. Ak sa napr. vysvetlenie chyby skladá z viacerých častí, bolo najviac špecifické vysvetlenie presunuté na začiatok a ďalej boli zobrazené všeobecnejšie definície. V pseudokóde to nebolo vždy pravidlom. Bolo tiež zabezpečené, aby všetky dáta typu \texttt{String} popisujúce chybu začínali malým písmenom, s jedinou výnimkou, a to ak by šlo o chybu vo veľkosti písmen.
    \chapter{Testovanie systému}
      \section{Jednotkové testy}
      \par Každá entita v každej vrstve aplikácie (pridať odkaz) je otestovaná jednotkovými testami, pričom testy jednotlivých vrstiev sú od seba oddelené za použitia Mock objektov (napísať, čo to je).
      \section{Integračné testy}
      \par mvn clean install -PintegrationTests-wildfly
      \section{Manuálne testovanie}
    \chapter{Nasadenie systému} \label{nasadenie}
      \section{Databázový systém}
      \par Ako databázový systém bol použitý PostgreSQL vo verzii XY. Prečo? Openstack cartrige? problem s LOB, ktorý je definovaný ako Long
      \section{Aplikačný server}
      \par Aplikačný server bol zvolený Wildfly vo verzii 9. Prečo? Rýchlosť, jednoduchá konfigurovateľnosť, opensource.
    \chapter{Záver}   
       \par Čo sa podarilo, čo sa nepodarilo a prečo, splnilo sa oficiálne zadanie?
       
           % Bibliography goes here
    \begin{thebibliography}{59}
    
    	\bibitem{crimson15}
  		Fahs, C. Ramsey.
  		\emph{EdX Overtakes Coursera in Number of Ivy League Partners}
  		[online].
  		2015
  		[cit. 2015-10-27].
  		Dostupné z: <\url{http://www.thecrimson.com/article/2015/10/2/edx-ivy-league-coursera/}>.

		\bibitem{comodo15}
  		Comodo CA Limited.
  		\emph{What is HTTPS}
  		[online].
  		2015
  		[cit. 2015-10-26].
  		Dostupné z: <\url{https://www.instantssl.com/ssl-certificate-products/https.html}>.
	
		\bibitem{openshift15}
  		Red Hat Inc.
  		\emph{Troubleshooting FAQs - How do I redirect traffic to HTTPS?}
  		[online].
  		2015
  		[cit. 2015-10-26].
  		Dostupné z: <\url{https://developers.openshift.com/en/troubleshooting-faq.html#_how_do_i_redirect_traffic_to_https}>.
  		
  		\bibitem{samwell14}
  		Samwell, Jon.
  		\emph{URL Route Authorization and Security in Angular}
  		[online].
  		2014
  		[cit. 2015-10-26].
  		Dostupné z: <\url{http://jonsamwell.com/url-route-authorization-and-security-in-angular/}>.
  		
  		\bibitem{yalkabov14}
  		Yalkabov, Sahat.
  		\emph{Build an Instagram clone with AngularJS, Satellizer, Node.js and MongoDB}
  		[online].
  		2014
  		[cit. 2015-10-26].
  		Dostupné z: <\url{https://hackhands.com/building-instagram-clone-angularjs-satellizer-nodejs-mongodb/}>.
  		
  		\bibitem{mik14}
  		Mikołajczyk, Michal.
  		\emph{Top 18 Most Common AngularJS Developer Mistakes}
  		[online].
  		2015
  		[cit. 2015-10-26].
  		Dostupné z: <\url{http://www.toptal.com/angular-js/top-18-most-common-angularjs-developer-mistakes}>.
  			
  	    \bibitem{sof1}
  		Stack Exchange Inc.
  		\emph{What is best way to store mp3 files in server ? Storing it in database (BLOB) , is right?}
  		[online].
  		2014
  		[cit. 2015-10-26].
  		Dostupné z: <\url{http://stackoverflow.com/questions/11958465/
  		what-is-best-way-to-store-mp3-files-in-server-storing-it-in-database-blob}>.
  		
  		\bibitem{rumanov12}
  		Rumanovský, Jakub.
  		\emph{Systém na trénovanie diktátov}
  		Brno, 
  		2012
  		Dostupné z: <\url{http://is.muni.cz/th/359581/fi_b/Bakalarka_FI_nal.pdf}>.
  		Bakalárska práca. Masarykova Univerzita.
  		
  		\bibitem{skvaril14}
  		Škvařil, Vojtěch.
  		\emph{Návrh algoritmu pro vyhodnocení bezkontextových pravopisných chyb}
  		Brno, 
  		2014
  		Dostupné z: <\url{http://is.muni.cz/th/399486/ff_b/BAKALARSKA_PRACE_27._6..pdf}>.
  		Bakalárska práca. Masarykova Univerzita.
  		
  		\bibitem{diffmatchpatch}
  		Google, Inc.
  		\emph{API Google-diff-match-patch}
  		[online].
  		2011
  		[cit. 2015-10-29].
  		Dostupné z: <\url{http://code.google.com/p/google-diff-match-patch/wiki/API}>.
  		
  		\bibitem{mozilla2015}
  		Mozilla Developer Network.
  		\emph{HTTP access control (CORS)}
  		[online].
  		2015
  		[cit. 2015-11-01].
  		Dostupné z: <\url{https://developer.mozilla.org/en-US/docs/Web/HTTP/Access_control_CORS}>.
  		
  		\bibitem{w3c2014}
  		W3C.
  		\emph{Cross-Origin Resource Sharing}
  		[online].
  		2014
  		[cit. 2015-11-01].
  		Dostupné z: <\url{http://www.w3.org/TR/cors/}>.
  		
  		\bibitem{sof2}
  		Stack Exchange Inc.
  		\emph{CORS angular js + restEasy on POST}
  		[online].
  		2014
  		[cit. 2015-11-01].
  		Dostupné z: <\url{http://stackoverflow.com/questions/22849972/cors-angular-js-resteasy-on-post}>.
		  	
	  	\bibitem{dzhuvinov15}
  		Dzhuvinov, Vladimir.
  		\emph{CORS Filter : Cross-Origin Resource Sharing for Your Java Web Apps}
  		[online].
  		2015
  		[cit. 2015-11-01].
  		Dostupné z: <\url{http://software.dzhuvinov.com/cors-filter-configuration.html}>.	  		
 		
 		\bibitem{sof3}
  		Stack Exchange Inc.
  		\emph{How to enable Cross domain requests on JAX-RS web services?}
  		[online].
  		2014
  		[cit. 2015-11-01].
  		Dostupné z: <\url{http://stackoverflow.com/questions/23450494/how-to-enable-cross-domain-requests-on-jax-rs-web-services}>.
 		
 		\bibitem{matei14}
  		Matei, Adrian.
  		\emph{How to add CORS support on the server side in Java with Jersey}
  		[online].
  		2014
  		[cit. 2015-11-01].
  		Dostupné z: <\url{http://www.codingpedia.org/ama/how-to-add-cors-support-on-the-server-side-in-java-with-jersey/}>.
  		
  		\bibitem{sof4}
  		Stack Exchange Inc.
  		\emph{When is it safe to enable CORS?}
  		[online].
  		2015
  		[cit. 2015-11-01].
  		Dostupné z: <\url{http://stackoverflow.com/questions/9713644/when-is-it-safe-to-enable-cors}>.
  		
  		\bibitem{spaghetti14}
  		Mitica, Gabrielle.
  		\emph{AngularJS and Basic Auth}
  		[online].
  		2014
  		[cit. 2015-11-01].
  		Dostupné z: <\url{http://spaghetti.io/cont/article/angularjs-and-basic-auth/12/1.html}>.
  		
  		\bibitem{watmore14}
  		Watmore, Jason.
  		\emph{AngularJS Basic HTTP Authentication Example}
  		[online].
  		2014
  		[cit. 2015-11-02].
  		Dostupné z: <\url{http://jasonwatmore.com/post/2014/05/26/AngularJS-Basic-HTTP-Authentication-Example.aspx}>.
  		
  		\bibitem{prasilova14}
  		Prášilová, Romana.
  		\emph{Návrh algoritmu pro vyhodnocení bezkontextových pravopisných chyb}
  		České Budějovice, 
  		2014
  		Dostupné z: <\url{http://theses.cz/id/29k3zb/DP-Prilov.pdf}>.
  		Diplomová práca. Jihočeská univerzita v Českých Budějovicích.
  		
  		\bibitem{majka}
		DOPLNIT CITACIU
  		Dostupné z: <\url{xyz}>.
  		
  		\bibitem{nlpfi}
  		DOPLN.
  		\emph{DOPLN}
  		[online].
  		2050
  		[cit. 2015-11-13].
  		Dostupné z: <\url{https://nlp.fi.muni.cz/projekty/ajka/tags.pdf}>.
  		
  		\bibitem{arlow11}
  		ARLOW, J., NEUSTADT
  		\emph{UML 2 a unifikovaný proces vývoje aplikací.}
  		[online].
  		I. 2011. 28 s.
		Computer Press. ISBN 978-80-251-1503-9. 
  		
  		http://goldfirestudios.com/blog/104/howler.js-Modern-Web-Audio-Javascript-Library
  		http://stackoverflow.com/questions/22684037/how-to-configure-wildfly-to-serve-static-content-like-images
  		https://blog.openshift.com/multipart-forms-and-file-uploads-with-tomcat-7/
  		http://stackoverflow.com/questions/32008182/wildfly-9-http-to-https
  		https://forums.openshift.com/changes-to-standalonexml-file
  		
  		  			
	\end{thebibliography} 
	
    \appendix
    \chapter{Používateľský manuál} 	  % Appendices

      \section{Rozhranie slúžiace na opravu diktátov} \label{navod_oprava}
      \par V nasledujúcej podkapitole bude popísané REST rozhranie spolu s korektným telom požiadavky a telom odpovede.
%tablebegin
\begin{table}[htbp]
\centering
\begin{tabular}{@{}|c|c|c|c|@{}}
\hline
\textbf{HTTP metóda} & \textbf{Cesta} & \textbf{Povolené role} & \textbf{Návratový kód}  \\ \hline
           POST     &  /api/correctDictate         &     neautorizovaný       &    201        \\ \hline
\end{tabular}
\caption{Popis rozhrania na opravu diktátu}
\label{rozhranie-oprava}
\end{table}
%tableend
%listingbegin
\lstset{language=Java}          % Set your language (you can change the language for each code-block optionally)
\begin{lstlisting}[frame=single]  % Start your code-block

{ 
	transcriptText: String,
	userText: String
}
\end{lstlisting}
\captionof{lstlisting}{Telo požiadavky}
%listingend
%listingbegin
\bigskip
\lstset{language=Java}          % Set your language (you can change the language for each code-block optionally)
\begin{lstlisting}[frame=single]  % Start your code-block

{ 
	totalMistakes: Integer,
	mistakes: 
		[{ 
		  id: Long,
		  mistakeCharPosInWord: Integer,
		  correctChars: String,
		  writtenChars: String,
		  correctWord: String,
		  writtenWord: String,
		  wordPosition: Integer,
		  lemma: String,
		  posTag: String,
		  sentence: String,
		  priority: Long,
		  mistakeType: String,
		  mistakeDescription: String 	
		}
		{
		...
		}]	
}
\end{lstlisting}
\captionof{lstlisting}{Telo odpovede}
%listingend
\bigskip
\par Viac o tom, čo jednotlivé atribúty znamenajú je uvedené v kapitole \ref{modul-diktaty}.


      \section{Manuál ku grafickému užívateľskému rozhraniu}
    \chapter{Snímky obrazoviek}
    \chapter{Zoznam jednotlivých typov chýb} \label{typy-chyb}
    \par Nižšie je uvedený zoznam možných typov chýb, ktorý bol jedným z výstupov bakalárskej práce Vojtěcha Škvařila\cite{skvaril14} v poradí, v akom sa vyskytol v priloženom pseudokóde popisujúcom jednotlivé pravidlá.
	\par \bigskip \textbf{Pravopis:}
\begin{itemize}
	\item Vyjmenovaná slova
	\item Psaní i/y po písmenu c
	\item Psaní předpon s-, z-
	\item Psaní předložek s, z (z postele, s knihou, s sebou)
	\item Pravopis a výslovnost přejatých slov se s – z
	\item Psaní dis-, dys-
	\item Psaní slov se zakončením -manie
	\item Psaní n – nn
	\item Psaní spřežek a spřahování
	\item Složená přídavná jména - První část přídavného jména je zakončena na -sko, -cko, -ně nebo –ově
	\item Velká písmena
\end{itemize}
\bigskip
	\textbf{Slovotvorba:}
\begin{itemize}
	\item Přídavná jména zakončená na -icí – -ící
	\item Přídavná jména zakončená na -ní – -ný
	\item Typ ačkoli – ačkoliv, kdokoli – kdokoliv
	\item Vokalizace předložek
\end{itemize}
\bigskip
	\textbf{Jevy, které nejsou v Akademické příručce českého jazyka přímo uvedeny:}
\begin{itemize}
	\item Zájmena typu vaši/vaší, ji/jí, ni/ní
	\item Psaní bě/bje, vě/vje, pě
	\item Souhlásky párové
	\item Diakritika
	\item i/í po měkkých a obojetných souhláskách
	\item Zájmena mně x mě a slova obsahující mě a mně
\end{itemize}

\par K týmto typom je naviac pridaný typ nadbytočného resp. chýbajúceho slova.
\par \bigskip
	\textbf{Pridané typy ktoré nie sú v bakalárskej práci uvedené:}
\begin{itemize}
	\item Chybějící slovo
	\item Nadbytečné slovo
	\item Ostatní
\end{itemize}
    
\end{document}
